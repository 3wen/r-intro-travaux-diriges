\usepackage[utf8]{inputenc} 
\usepackage[T1]{fontenc} % pour taper les lettres accentuees
\usepackage[english]{babel}
\usepackage{amsthm,amssymb,amsbsy,amsmath,amsfonts,amssymb,amscd,mathrsfs}
% Mise en page
\usepackage{geometry} %pour la modification des marges
\usepackage{fancyhdr} %pour modification des pieds de page
\usepackage{lastpage} %numero de la derniËre page
\usepackage{lscape} % Pour pouvoir activer le mode landscape
\usepackage{lmodern}
%Images, figures, etc.
\usepackage{booktabs} %tableaux
\usepackage{float}	%pour forcer le placement des images.
\usepackage{graphicx} %pour afficher des images
\usepackage{longtable} %tables sur plusieurs pages
\usepackage{animate} %transformer des gifs
\usepackage{caption}

\def\frenchtablename{Tableau}

% Chemin vers les images
\graphicspath{{../figs}}
%\usepackage{subcaption}
\captionsetup{labelformat=simple}
\usepackage[scriptsize]{subfigure}
\usepackage{multirow} 			%fusionner lignes
\usepackage{tikz}
\usepackage{adjustbox}
\usetikzlibrary{arrows,positioning}
\usetikzlibrary{mindmap,trees,shadows,backgrounds}
\usepackage{tabularx}
%Code
\usepackage{verbatim}%insertion de code
\usepackage{listings}
%Polices, format,couleurs
\usepackage{dsfont} % Pour les lettres mathematiques

% Bibliographie
\usepackage[nottoc]{tocbibind}
\usepackage{hyperref} %pour que les references soient des liens hypertextes
\usepackage{natbib}
\usepackage{bibentry}
%\usepackage{color}
\usepackage{xcolor,colortbl}
\usepackage{ragged2e} % Pour justifier
%Symboles, theoremes
\newcommand{\iid}{\stackrel{\mathrm{iid}}{\sim}}
\newtheorem{theorem}{Theorem}[part]
\newtheorem{lemma}[theorem]{Lemma}
\newtheorem{hypothese}{Hypoth\`ese}
\newtheorem{corollary}[theorem]{Corollary}
\usepackage{enumerate}%listes
\usepackage[subnum]{cases}% cas numerotes
\newtheorem{rmarque}{Remarque}[section]
\newtheorem{exmp}{Exemple}[section]
% Numeroter les equations en fonction du numero de section
\numberwithin{equation}{section}
\usepackage[official,right]{eurosym} % symbols euro
\usepackage{gensymb} % symbole degre \degre
\usepackage{rotating}
\usepackage{multicol}
\usepackage{empheq}
% Bibliographie
\renewcommand*{\bibfont}{\scriptsize} % Pour avoir la biblio en plus petit



\justifying %on justifie le texte du document


% Pour aligner verticalement du texte dans des latbleaux LaTeX
\usepackage{array,dcolumn}
\newcolumntype{C}[1]{>{\centering\arraybackslash}m{#1}}
\newcolumntype{R}[1]{>{\raggedleft\arraybackslash}m{#1}}
\newcolumntype{L}[1]{>{\raggedright\arraybackslash}m{#1}}


% -------- %
% Couleurs %
% -------- %
\definecolor{light-gray}{gray}{0.85}
\definecolor{vert}{RGB}{214, 245, 214}
\definecolor{gris}{RGB}{230, 247, 255}
\definecolor{bleu}{RGB}{163, 192, 230}
\definecolor{grisfonce}{RGB}{220,220,220}
\definecolor{jaune}{RGB}{255, 243, 199}
\definecolor{rouge}{RGB}{207, 47, 68}
\definecolor{oran}{RGB}{255, 161, 0}
% Couleurs distinctes pour daltoniens et daltoniennes
\definecolor{wongBlack}{HTML}{000000}
\definecolor{wongGold}{HTML}{E69F00}
\definecolor{wongLightBlue}{HTML}{56B4E9}
\definecolor{wongGreen}{HTML}{009E73}
\definecolor{wongYellow}{HTML}{F0E442}
\definecolor{wongBlue}{HTML}{0072B2}
\definecolor{wongOrange}{HTML}{D55E00}
\definecolor{wongPurple}{HTML}{CC79A7}
% Deuxieme set de couleurs pour daltoniens et daltoniennes
\definecolor{IBMBlue}{HTML}{648FFF}
\definecolor{IBMPurple}{HTML}{785EF0}
\definecolor{IBMMagenta}{HTML}{DC267F}
\definecolor{IBMOrange}{HTML}{FE6100}
\definecolor{IBMYellow}{HTML}{FFB000}

% Commandes pour ecrire en gras et en couleur
\newcommand*\grasO[1]{\textbf{\textcolor{wongOrange}{#1}}} 
\newcommand*\grasR[1]{\textbf{\textcolor{rouge}{#1}}}
\newcommand*\grasV[1]{\textbf{\textcolor{wongGreen}{#1}}} 
\newcommand*\grasB[1]{\textbf{\textcolor{wongBlue}{#1}}} 
\newcommand*\grasJ[1]{\textbf{\textcolor{wongYellow}{#1}}} 
\newcommand*\grasP[1]{\textbf{\textcolor{wongPurple}{#1}}} 

% ----------------- %
% Boites de couleur %
% ----------------- %
\usepackage{tcolorbox}



\makeatletter
\let\@@magyar@captionfix\relax
\makeatother

% Compteur pour les blocs de remarques
\newcounter{countremarque}

\newenvironment{remarque}{%
 \refstepcounter{countremarque}
    \begin{tcolorbox}[width=\linewidth, colback=blue!3, boxrule=0.5pt,arc=0pt,title = Remarque \thecountremarque]
    }{\end{tcolorbox}}
\numberwithin{countremarque}{section}


\usepackage{color}
\usepackage{framed}
\setlength{\fboxsep}{.8em}
\setlength{\headheight}{14.5pt}

% Cadre de note (notebox)
\newenvironment{notebox}{
	\begin{tcolorbox}[colback=jaune,coltext=black,colframe=grisfonce]}{\end{tcolorbox}
}

% Cadre de note (greenbox)
\newenvironment{greenbox}{
	\begin{tcolorbox}[colback=vert,coltext=black,colframe=grisfonce]}{\end{tcolorbox}
}

% Cadre de note (bluebox)
\newenvironment{bluebox}{
	\begin{tcolorbox}[colback=bleu,coltext=black,colframe=grisfonce]}{\end{tcolorbox}
}

% Cadre de note (redbox)
\newenvironment{redbox}{
	\begin{tcolorbox}[colback=rouge,coltext=white,colframe=grisfonce]}{\end{tcolorbox}
}

% Cadre de note (orangebox)
\newenvironment{orangebox}{
	\begin{tcolorbox}[colback=oran,coltext=white,colframe=grisfonce]}{\end{tcolorbox}
}


% Paramètres des liens hypertextes dans le document
\hypersetup{
bookmarks=true,
pdfauthor={},
unicode=false,
pdftoolbar=true,
pdfmenubar=true,
pdffitwindow=false
pdfnewwindow=true,
colorlinks=true,
linkcolor=wongBlue,
citecolor=wongBlue,
filecolor=wongBlue,
urlcolor=wongBlue}
